\documentclass[a4paper,11pt]{article}

\usepackage{kotex}
\usepackage{amsmath,amsfonts,amssymb}
\author{미시경제이론 수강 학생들}

\title{김영산, 왕규호 미시경제학 오류 리스트}
\begin{document}

\maketitle

\section{Chapter 3} % (fold)
\label{sec:chapter_3}
\begin{itemize}
	\item [p.70] Line 9: 예산선은 $(0,m/p_1)$을 축으로 하여 (틀림) (2018sp 양상준)
	\begin{quote}
		예산선은 $(0,m/p_2)$를 축으로 하여
	\end{quote}
	\item [p.80] 그림 3-11: 축의 $w_1,w_2$가 반대로 표시됨. 즉, 세로축이 $w_2$, 가로축이 $w_1$이어야 함 (2018sp 김정은)
\end{itemize}
% section chapter_3 (end)

\section{Chapter 4} % (fold)

\label{sec:ch4}
\begin{itemize}
	\item [p.87] Line 5: $y$는 $x$보다 약선호되면 (틀림) (2018sp 임가영)
	\begin{quote}
		$y$는 $z$보다 약선호되면
	\end{quote}
	\item [p.115] 수식오류 (2018sp 김정은)
	\[
		V(x_1,x_2)=a \ln x_1 + b\ln x_2
	\]
	\item [p.129] 연습문제 7번 2)문제 (2018sp 김정은)
	\[
		U(x_1,x_2) = \sqrt{x_1x_2}
	\]
\end{itemize}
% section ch4 (end)

\section{Chapter 5} % (fold)
\label{sec:chapter_5}
\begin{itemize}
	\item [p.136] Line 4: $x_2^\ast=x_1(p_2) $ (2018sp 임가영)
	\[
		x_2^\ast=x_2(p_2)
	\]
	\item [p.193] Line 3: 시장수요함수 오류 (2018sp 정연진)
	\[
		\text{시장수요함수: } x_1^M = x(p_1:p_2,p_3,m_1,m_2,m_3)
	\]
\end{itemize}
% section chapter_5 (end)

\section{Chapter 6} % (fold)
\label{sec:chapter_6}
\begin{itemize}
	\item [p.214] Line 10: 보상변화 정의부분 $CV=m_0 - e(p_1^1,p_2^0,u_0)$ (틀림) (2018sp 김동현)
	\[
		CV= e(p_1^1,p_2^0,u_0) - m_0
	\]
	참고로, 본문중의 $m_0-m_1$ 은 오류가 아님. $m_0-m_1>0$ 만큼 "감소"한다는 것이므로 $m_1-m_0<0$ 만큼 증가한다는 말과 같은 뜻임. 
\end{itemize}
% section chapter_6 (end)

\section{Chapter 11} % (fold)
\label{sec:chapter_11}
\begin{itemize}
	\item [p.378] Line -1(마지막줄): 자본의 평균생산 (2018sp 강우진)
	\[
		AP_K (L,K) = \frac{F(L,K)}{K}
	\]
	\item [p.386] 그림 11-9 $F(L,K)=q_1$ (틀림) (2018sp 강우진)
	\[
		F(L,K)=q_0
	\]
\end{itemize}
% section chapter_11 (end)
\end{document}